\section{Exercises}

\begin{enumerate}
\item
Write and TEST Prolog definitions for the following predicates. You
will be graded on the simplicity and elegance of your solutions, AND
how well you test them.  You can download XSB from
xsb.sourceforge.net for unix (for which you must configure and
compile, according to instructions) or for Windows (as a zip file that
unzips to a set of directories with an executable in
XSB/config/i686-pc-cygwin/bin/xsb.exe).  Contact me if you need help.

\begin{enumerate}
\item
member(Element,List) iff Element is a member of the list List. \\
Give a query that uses it to produce all members of a list that are
greater than some integer.

\item
last(Element,List) iff Element is the last element of list List.

\item

consecutive(X,Y,L) iff X and Y are consecutive elements (in that order) of list L.

\item
delete\_first(X,Li,Lo) iff list Lo is list Li with the first
occurrence of X removed.

\item
delete\_all(X,Li,Lo) iff list Lo is list Li with all occurrences of X
removed.

\item
append(L1,L2,L3) iff list L3 is the result of concatenating list L1
and L2. ***Use it to split a list into all pairs of sublists that
concatenate to produce the given list.

\item
sublist(Sl,L) iff list Sl is a (contiguous) sublist of list L.

\item
\begin{enumerate}
\item
intersect(L1,L2,I) iff I is the list that is the intersection of the
lists L1 and L2. (Assume L1 and L2 do not have duplicates within
themselves but may be in any order.  Produce in I one list that
contains all the elements in any order that are in both L1 and L2.)

\item
intersect(L1,L2,I) iff I is the list that is the intersection of
the lists L1 and L2. (Assume L1 and L2 are in increasing order (using
@<).  The list representing the intersection should also be in order.)
\end{enumerate}

\item
reverse(L,R) iff list R is the reverse of the list L.  What is
the complexity of your solution for a list of length n?  Explain.

\item
palindrome(L) iff the list L is a palindrome, i.e. reads the same
backwards and forwards.

\item
permutation(L1,L2) iff list L2 is a permutation of list L1. (It
should be able to generate all permutations.)

\item
split(E,L,P1,P2) iff lists P1 and P2 are a partition of list L of
integers, such that all elements of P1 are less than or equal E and
all elements of P2 are greater than E.

\item
quick\_sort(L,S) iff S is the sorted version of the list L of
integers. The idea of quicksort is 1) to choose an element from the
list; 2) split the list into two lists putting all elements smaller
than (or equal to) the chosen element into the first list and the
others into the second (Hint: Use split/4 for this); 3) recursively
sort the two sublists; and 4) concatenate the two sorted sublists
(putting the chosen element back in the middle).
\end{enumerate}

\item
Consider the very simple "English" grammar we discussed in the chapter:

\begin{verbatim}
s --> np,vp.
np --> det,n.
vp --> tv,np.
vp --> v.
det --> [the].
det --> [a].
det --> [every].
n --> [man].
n --> [woman].
n --> [park].
tv --> [loves].
tv --> [likes].
v --> [walks].
\end{verbatim}

Extend this grammar to include some prepositional phrases, as "in the
park".  Also include several proper names, such as john, mary, and
rohit.  You should add them to the grammar in a natural way so that
they follow the (simple syntactic) rules of English grammar.  For
example, prepositional phrases can modify nouns and they can modify
verb phrases.

Example sentences that you should be able to process include:

\begin{enumerate}
\item
a man walks to the park
\item
the man in the park likes mary
\item
the man saw a woman in the park with a telescope
\item 
the man in the park with a telescope walks
\item 
the man walks in the park with a telescope 
\item 
many, many more, some of which may not make much sense but are ``syntactically'' OK.
\end{enumerate}
(Add the necessary words to their appropriate categories.)

Test your grammar on these examples and others of your own choosing.

\item
Extend your grammar of the previous problem to construct a parse tree for each
sentence it recognizes.

Run your parser to construct parse trees for each of the example
sentences of problem 1, and for several others of your own choosing.
Does your parser produce multiple different parses for some of your
examples?  Should it?  Discuss.

\end{enumerate}

\section{Exercise Discussion}

The usual definition of reverse, using append (and called naive
reverse) does not work for all calling modes.  It only uses
unification so why doesn't it work in all calling modes as append/3
does?  Consider the following definition:
\begin{verbatim}
reverse(L,R) :- same_length_list(L,R), reverse1(L,R).

same_length_list([],[]).
same_length_list([_|L1],[_|L2]) :- same_length_list(L1,L2).

reverse1([],[]).
reverse1([X|L],R) :- reverse1(L,Rhs), append(Rhs,[X],R).
\end{verbatim}
reverse1/2 is the standard definition of naive reverse.  This reverse
works in all directions.  Even:
\begin{verbatim}
| ?- reverse(L,R).
\end{verbatim}
gives correct answers.
We could also use the linear version of reverse1, with an accumulator:
\begin{verbatim}
reverse1([],R,R).
reverse1([X|L],S,R) :- reverse1(L,[X|S],R).
\end{verbatim}
and calling reverse1(L,[],R) in the body of reverse.
(Note that same\_length\_list is done only once and is linear, so this
definition of reverse has the complexity of the reverse1 definition.)

To come up with this definition, I was thinking of the definition of
naive reverse and why we couldn't interchange the two calls in the
body of the recursive rule and put the append first.  And the problem
is that with the direct call (with first argument bound and second a
variable) we get a call of the form
\begin{verbatim}
   append(-,[+],-)
\end{verbatim}
(where - indicate variables and + indicates a bound value.)  But there
are infinitely many pairs of lists that stand in this relationship, so
Prolog is necessarily in an infinite loop.  So why do we need to look
at only finitely many of these lists?  Because we know the length of
the resulting list.  So by setting the resulting list to be of the
right length (the same as the input list), we can make that call have
only finitely many (in fact, only 1) solutions.  So that's how I came
to this definition.  Which also points out that for the naive
definition, we might as well interchange the body calls and actually
do the append first, which will make the definition tail recursive and
save space.  Of course its time complexity is still quadratic
